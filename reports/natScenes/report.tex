% @Author: Athul Vijayan
% @Date:   2014-05-09 13:56:20
% @Last Modified by:   Athul
% @Last Modified time: 2016-02-01 10:35:02

\documentclass[11pt,paper=a4,answers]{exam}
\usepackage{graphicx,lastpage}
\usepackage{subcaption}
\usepackage[table]{xcolor}

\usepackage{multirow}
\usepackage{upgreek}
% \usepackage{float}
\usepackage{placeins}
\usepackage[bookmarks]{hyperref}
\usepackage{censor}
\usepackage{amsmath}
\usepackage{amssymb, amsthm}
\newcommand{\argmax}{\operatornamewithlimits{argmax}}
\usepackage{bm}
\usepackage{caption}
\usepackage{morefloats}
\usepackage{enumerate}


\newcommand{\cb}[1]{{\cellcolor{black! 15 }$ #1$}}
\newcommand{\Oh}{\bm{\mathcal{O}}}
\newcommand{\cw}[1]{{\cellcolor{black! 35 }$ \color{white} #1$}}
\censorruledepth=-.2ex
\censorruleheight=.1ex
\hyphenpenalty 10000
\usepackage[paperheight=10.5in,paperwidth=8.27in,bindingoffset=0in,left=0.8in,right=1in,
top=0.7in,bottom=1in,headsep=.5\baselineskip]{geometry}
\flushbottom
\usepackage[normalem]{ulem}
\newcommand{\E}{\mathbb{E}}
\renewcommand\ULthickness{2pt}   %%---> For changing thickness of underline
\setlength\ULdepth{1.5ex}%\maxdimen ---> For changing depth of underline
\renewcommand{\baselinestretch}{1}
\pagestyle{empty}
\renewcommand{\vec}[1]{\mathbf{#1}}
\pagestyle{headandfoot}
\headrule

\newcommand{\continuedmessage}{%
\ifcontinuation{\footnotesize continues\ldots}{}%
 }
\newcommand{\e}{\mathrm{e}}
\renewcommand{\v}{\mathrm{v}}
\newcommand{\x}{\mathrm{x}}
\newcommand{\cov}{\mathrm{cov}}
\newcommand{\q}[1]{\ensuremath{\mathit{q}^{-#1}}}
\runningheader{\footnotesize \today}
{\footnotesize Neural Data}
{\footnotesize Page \thepage\ of \numpages}
\footrule
\footer{\footnotesize}
{}
{\ifincomplete{\footnotesize section \IncompleteQuestion\ continues
on the next page\ldots}{\iflastpage{\footnotesize End}{\footnotesize Please go        on to the next page\ldots}}}

\usepackage{cleveref}
\crefname{figure}{figure}{figures}
\crefname{question}{question}{questions}
%==============================================================
\begin{document}
\noindent
\begin{minipage}[l]{.1\textwidth}%
\noindent
\end{minipage}
\hfill
\begin{minipage}[r]{.68\textwidth}%
\begin{center}
{\large \bfseries \par
\Large Neural Data Analysis \\[2pt]
\vspace{6pt}
\small   \par}
\end{center}
\end{minipage}
\begin{minipage}[l]{.195\textwidth}%
\noindent
{\footnotesize}
\end{minipage}
\par
\noindent
\uline{DONlab \hfill \normalsize\emph \hfill    Athul Vijayan(ED11B004)}\\

\tableofcontents
\newpage
\section{Introduction} % (fold)
\label{sec:introduction}
This gives background.
% section introduction (end)

\section{Experiment} % (fold)
\label{sec:experiment}
\subsection{Setup} % (fold)
\label{sub:setup}
Calcium images were scanned using a Two-photon calcium imaging device on mice injected with GCaMP6f. Firing rates of these cells were then inferred using a fast nonnegative deconvolution algorithm on the Calcium imaging data.\\


\textbf{Visual stimuli}\\
Creation of noise movies. We developed an algorithm that allowed us to
create noise images with a user-defined spectral slope. To do so, we took
advantage of the inverse-square law: $P \sim k^{-\alpha}$ , which translates to a circle
with radius $\alpha$ in two-dimensional Fourier space. Thus, we constructed all
noise movies in the Fourier domain. We first defined a matrix of the same
size as the original image (256 $\times$ 256 pixels) and then created a noise
amplitude spectrum as a 2D circle of radius $\alpha$ , with $\alpha$ taking values from
0 (K0 movie) to $\sqrt{2}$ (K2 movie). This was due to the squared relationship
between the amplitude spectrum and the power spectrum. To create the
final noise image, we combined this noise amplitude spectrum with a
random phase spectrum, where phase values were randomly sampled
from the range $0 - 2\pi$ . The final noise images were visualized by comput-
ing its 2D inverse Fourier transform. Each frame of the noise movie was
created using a new random seed, and as a result, the raw noise movies
had no temporal correlations between frames.
Noise-masking procedure. Figure \ref{haha} provides a schematic of the noise-
masking procedure. First, each frame of a natural movie was decomposed into its Fourier components (phase and amplitude) via a 2D fast Fourier
transform implemented in MATLAB. Next, a noise image was created as
described above. The phase spectrum of the original movie was then
combined with the amplitude spectrum of the noise movie. The resulting
image was then inverse Fourier transformed to yield a noise-masked
movie frame. This procedure was repeated for all frames. We used a total
of five different natural movies, each 4 s in duration, from the van Hat-
eren movie database

% subsection setup (end)
\subsection{Data} % (fold)
\label{sub:data}
MATLAB datafile AmpMov.mat contains the following fields.\\
Experiments are done various days, the data corresponding to each day are in each folder.\\
Subscript \_nat corresponds to responses to video stimuli for original natural scenes video. similarly subscripts \_K0, \_K\_1, \_K1\_5, \_K\_2 denote responses to manipulated natural scenes video stimuli.
\begin{center}
    \begin{tabular}{|l|l|}
        \hline
        Field                                 & Description \\
        \hline
        Sorted.SpikeRate                  & blah blah \\
        Blank                             & Dimension 47 x 16800 \\
        NumNeurons                        & Number of neurons sampled \\
        NumMovies                         & Number of movies used as stimulus \\
        M\_nat                            & Dimension 4 x 47 x 1200 \\
        MT\_nat                           & Dimension 4 x 47 x 200 x 6 \\
        MTA\_nat                          & Dimension 4 x 47 x 200 \\
        MTNA\_nat                         & Dimension 4 x 47 x 1200 \\
        \hline    
    \end{tabular}
\end{center}

% subsection data (end)
% section experiment (end)

\section{Study of Reliability} % (fold)
\label{sec:study_of_reliability}

% section study_of_reliability (end)

\section{Study of correlation} % (fold)
\label{sec:study_of_correlation}

% section study_of_correlation (end)

\section{Searching for Motifs} % (fold)
\label{sec:searching_for_motifs}

% section searching_for_motifs (end)
% ============================== Content ends here
\end{document}